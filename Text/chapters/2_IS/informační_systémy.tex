\section{Informační systémy}
%%%%%%%%%%%%%%%%%%%%%%%%%%%%
Před samotnou definicí informačního systému je vhodné nejdříve definovat pojmy \textit{informace} a \textit{systém}. To nám pomůže lépe pochopit samotnou definici informačních systémů a role, které tyto systémy hrají.
%%%%%%%%%%%%%%%%%%%%%%%%%%%%
\subsection{Informace}
Slovo \textit{informace} má velmi široký význam, proto je dobré začít obecně. Pro takové pojetí se nabízí Wienerova definice informace:
\begin{definition}
Informace je pojmenování pro obsah toho, co se vymění s vnějším světem, když se mu přizpůsobujeme a působíme na něj svým přizpůsobováním. \cite{Wiener1963}
\end{definition}
Tato definice je velmi obecná ale zachycuje dva důležité aspekty informace. Prvním je obsah informace a druhým je způsob výměny tohoto obsahu. 

Způsob výměny nám pomůže tuto definici lépe pochopit. Člověk je vybaven pěti základními smysly, pomocí kterých vnímáme vnější svět a přijímáme obsah informací.

Tím se pomalu dostáváme k informačním systémům, které, jak později formálně definujeme, s informacemi pracují a poskytují je svým uživatelům podle jejich potřeb.
%%%%%%%%%%%%%%%%%%%%%%%%%%%%
\subsection{Systém}
Zastavme se ještě u definice systému. Ta se mezi některými autory mírně liší, a to hlavně v hloubce, ve které pojem definují. \cite{Tvrdikova2008}\cite{Pour2015}\cite{Vymetal2009} Na nejobecnější úrovni se ale shodují na definici:
\begin{definition}
Systém je množina prvků a vazeb mezi nimi.
\end{definition}

Jednoduchý příklad z běžného života nabízí ve své knize Pour, a to školu. Jednotlivé prvky ve škole tvoří například studenti, studijní obory a předměty. Vztahy mezi těmito prvky představují zmíněné vazby. \cite{Pour2015}

Pojem ve své knize dále přibližuje Tvrdíková. Ta rozděluje systémy na přirozené a umělé. 

Přirozené systémy jsou ty, které vznikly bez zásahu člověka, naopak systémy umělé jsou člověkem vytvořené. \cite{Tvrdikova2008}

Z tohoto pohledu patří informační systémy mezi systémy umělé, které jsou člověkem výrazně ovlivněny. Proto, pro správné fungování těchto systémů je potřeba s nimi správně zacházet.
%%%%%%%%%%%%%%%%%%%%%%%%%%%%
\subsection{Informační systém}
Konečně se dostáváme k definici informačního systému. I zde se různé zdroje ve svých definicích mírně liší. \cite{Tvrdikova2008}\cite{Pour2015}\cite{Vymetal2009} Využijme definici, kterou poskytuje Tvrdíková:
\begin{definition}
Informační systém je soubor lidí, metod a technických prostředků zajišťujících sběr, přenos, uchování, zpracovaná a prezentaci dat s cílem tvorby a poskytovaní informací dle potřeb příjemců informací činných v systémech řízení. \cite{Tvrdikova2008}
\end{definition}

Všimněme si, že definice v sobě zahrnuje i člověka, ten je důležitou a nedílnou součástí těchto systémů. 

Dále si můžeme všimnout, že v definici informačního systému není žádná zmínka o počítačových systémech. Není to náhodou, mezi technické prostředky informačních systémů patří i klasická tužka a papír.

Zaměřme se na informační systémy opřené o moderní počítače a informační technologie. \mbox{Tvrdíková}~\cite{Tvrdikova2008} představuje rozbor takovýchto systémů na jednotlivé komponenty:

\begin{itemize}
    \item \textbf{technické prostředky (hardware)} -- počítačové systémy a jejich periferie,
    \item \textbf{programové prostředky (software)} -- systémové programy řidící chod počítače,
    \item \textbf{organizační prostředky (orgware)} -- pravidla pro provoz IS,
    \item \textbf{lidská složka (peopleware)} -- řešení adaptace a účiné obsluhy IS v počítačovém prostředí,
    \item \textbf{reálný svět} -- informační zdroje, legistalitva, normy.
\end{itemize}

Pro vytvoření kvalitního a efektivního informačního systému je důležité nezanedbat žádný z těchto komponentů.
