\section{Využití CRM systému v oblasti pneuservisů}
%%%%%%%%%%%%%%%%%%%%%%%%%%%%
Většina služeb pneuservisů jsou jednorázového charakteru. Mezi tyto služby patří přezutí kol, vyvážení pneumatik, opravy pneumatik, prodej pneumatik nebo disků a ekologická likvidace. Ačkoli jsou tyto interakce jednorázové, tak je může zákazník využívat opakovaně. 

Pneuservisy samozřejmě cílí na opakované návštěvy. Například službou uskladnění pneumatik, kterou se snaží zachytit a přesvědčit zákazníky, aby se k nim v další sezóně vrátili. Tato služba samozřejmě již vyžaduje vést nějakou evidenci minimálně toho co je uloženo, od kdy a komu to patří.

Po krátkém zamyšlení je jasné, jak mohou CRM systémy agendu pneuservisu podporovat. Evidence zákazníků, automobilů, pneumatik, disků, služeb, atd. může být základem nejen pro operativu, ale i pro další taktické a strategické rozhodování.

Od nejnižší úrovně, tyto systémy podporují různé aspekty provozu. Přehled v datech může vést k přehlednější evidenci uskladnění. Automatizace, kterou CRM systémy podporují, může zrychlit a ulehčit administrativní práci. Systém může být základem pro vytvoření webového rozhraní pro objednávání služeb.

Na úrovni taktické, většina dostupných CRM systémů podporuje vytváření marketingových kampaní, které pomáhají zákazníky do pneuservisu nalákat.

Moderní CRM systémy také obsahují reportovací funkce, které mohou poskytovat důležité statistiky pro podporu strategického rozhodování.

CRM systémy jsou pro pneuservisy jasnou volbou, a to hlavně kvůli častým interakcím se zákazníky.
