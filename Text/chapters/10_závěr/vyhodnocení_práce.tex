\section{Vyhodnocení práce a splnění cílů}
%%%%%%%%%%%%%%%%%%%%%%%%%%%%
Hlavním cílem bakalářské práce bylo provést analýzu a implementaci informačního systému pro pneuservis Láva Design na platformě Salesforce. Pro splnění tohoto cíle bylo potřeba práci dekomponovat na teoretickou a praktickou část.

V teoretické části jsem provedl rešerši v oblasti informačních systémů, definoval jsem informační systémy a popsal jednotlivé typy IS a jejich využití. Dále jsem se zaměřil na CRM systémy, porovnal možnosti, které jsou na trhu dostupné a zaměřil se na společnost Salesforce, Inc. 

V další kapitole jsem se věnoval obecné agendě pneuservisů a popsal, jak by mohly CRM systémy tuto agendu podpořit.

Praktická část se skládala z běžných fázi vývoje softwaru, tedy analýzy a sběru požadavků, návrhu, implementace a testování. K této části jsem přidal kapitolu ekonomicko-manažerského vyhodnocení. 

Na základě poznatků z teoretické části a společné schůze s panem Lávičkou jsem provedl analýzu podniku Láva Design a posbíral požadavky na nový IS. 

Poté následoval návrh s pomocí nástrojů softwarového vývoje, konkrétně diagramu Use Case a doménového modelu. V této části mi nejvíce času zabral doménový model, který sloužil jako odrazový můstek pro implementaci.

Po dohodě s vedoucím byla implementace rozdělena na dvě etapy. Cílem tohoto rozdělení bylo v první etapě vytvořit první verzi komplexního systému, který je možné využít i v dalších pneuservisech.

Na základě návrhu jsem provedl implementaci na platformě Salesforce, kterou jsem pokryl veškeré případy užití první etapy.

Ve fázi testování jsem se zaměřil na uživatelské testy. Nejdříve jsem se pokusil přesunout pár záznamů ze starého excelu do nového IS. Při tomto testu jsem odhalil nedostatky implementace, které jsem opravil. Systém byl uživatelsky otestován i dobrovolníky, ale během tohoto testování nebyly nalezeny další chyby.

Poté jsem systém a práci v něm představil vedoucímu práce a IT expertovi a upravil systém podle výhrad.

Nakonec jsem provedl ekonomicko-manažerské vyhodnocení, kde jsem se zaměřil na náklady současného systému i nového systému, popsal přínosy nového IS a výhled pokračování do budoucna.

Tímto posledním krokem byly splněny všechny cíle, které si práce určila.
