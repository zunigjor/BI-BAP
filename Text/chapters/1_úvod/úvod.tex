\chapter{Úvod}
%%%%%%%%%%%%%%%%%%%%%%%%%%%%
Během pandemie COVID-19 si velká část z nás uvědomila, jak je digitalizace důležitá. Většina malých podniků se na svém začátku obejde s klasickými nástroji, jako jsou například tužka a papír nebo jednoduché dokumenty z prostředí MS Office. 

Zmíněná pandemie nebo růst podniku jsou jen některé z faktorů, které z jednoduchých nástrojů, dělají nástroje nedostatečné. Pro malé podniky je digitalizace a systematizace důležitým krokem v jejich dalším růstu. 

Jedním z těchto podniků je i pneuservis Láva Design, který provozuje pan Petr Lávička sám. Ten se potýká s rostoucím zájmem zákazníků a jejich telefonáty, hlavně během přezouvacích sezón, ho ruší od práce. 

Informace od zákazníků si nyní zapisuje do zápisníků, a na konci každé sezóny, je přepisuje do excelové tabulky. To vede ke ztrátě času přepisováním a nepořádku v datech.

Možným řešením těchto problémů se nabízí informační systém, který by panu Lávičkovi ušetřil čas a uvolnil ruce.

Sám pocházím z rodiny podnikatelů, a vím jak jsou informační systémy pro malé a rostoucí podniky potřebné. To mi dodalo motivaci provést analýzu tohoto podniku a návrh a implementaci CRM systému pro tento pneuservis.
%%%%%%%%%%%%%%%%%%%%%%%%%%%%
\section{Cíl práce}
Hlavním cílem práce je návrh a implementace první verze informačního systému pro pneuservis Láva Design. 

Pro splnění tohoto cíle je potřeba nejdříve provést rešerši v oblasti informačních systémů, seznámit se s obecnou agendou pneuservisů a popsat jak mohou IS tuto agendu podporovat.

Před návrhem je potřeba provést analýzu pneuservisu Láva Design a posbírat požadavky majitele. Až poté je možné vytvořit návrh a implementaci, kterou je potřeba otestovat.

Na konec je potřeba provést ekonomicko-manažerské vyhodnocení přínosů a nákladů nového IS a výhled budoucího vývoje.
%%%%%%%%%%%%%%%%%%%%%%%%%%%%
\section{Struktura práce}
Práce se nejdříve věnuje rešerši v oblasti informačních systémů, obecnou agendou pneuservisů a využitím CRM systémů v této oblasti. 

Poté se věnuje analýze pneuservisu Láva Design a sběru požadavků na řešení. Práce pokračuje návrhem nového systému, implementaci první verze IS a uživatelskému testování nového systému. 

V poslední části je provedeno ekonomicko-manažerské vyhodnocení a výhled do budoucna.
