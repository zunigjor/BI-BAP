\section{Náklady a přínosy}
Náklady a přínosy nového systému lze těžko vyhodnotit, protože systém ještě ani není předán, to ale nebrání provedení odhadů.
%%%%%%%%%%%%%%%%%%%%%%%%%%%%
\subsection{Náklady současného stavu}
Náklady současného stavu jsou hlavně časové a týkají se převážně dvou oblastí, telefonátů od zákazníků a přepisu informací do excelu. Další náklad jsou peníze o které pneuservis přichází kvůli času tráveném udržování tohoto systému.

Prvním časovým nákladem jsou telefonáty od zákazníků a zapisování informací do zápisníku. Odhadem, pokud během jedné sezóny pneuservis provede okolo 300 výměn, na každou z těchto výměn se musí zákazníci telefonicky objednat. Pokud telefonáty, včetně zápisu do zápisníku, trvají kolem 5 minut. Můžeme odvodit, že čas strávený těmito telefonáty se pohybuje kolem 1500 minut (25 hodin) za sezónu.

Druhý časový náklad je přepis informací ze zápisníku do excelové tabulky. Opět využijme 300 výměn, vyhledání informace v zápisníku a zápis do excelu může trvat v okolí 5 minut. Tedy dalších 1500 minut (25 hodin) každou sezónu.

Pokud obsloužení jednoho zákazníka, trvá kolem 30 až 60 minut. Můžeme odvodit, že každá ušetřená hodina se rovná jednomu až dvěma zákazníkům.

Celkem, čas strávený prací se současným systémem se pohybuje v okolí \textbf{50 hodin za sezónu}.
%%%%%%%%%%%%%%%%%%%%%%%%%%%%
\subsection{Náklady nasazení nového IS} \label{NakladyNasazeni}
Náklady na nasazení nového IS jsou finanční i časové. Skládají se z nákladů za služby Salesforce, nákladů na instalaci aplikace a školení.

Pro používání služeb Salesforce je potřeba pneuservis registrovat a vybrat edici. Kvůli využívání vlastních objektů je edice Professional ideálním výběrem. Cena této edice je \texteuro{75} za uživatele za měsíc. Platba za tuto edici se provádí na rok dopředu, tedy celková cena, pro jednoho uživatele, je \texteuro{900}. Při kurzu 25 Kč za Euro se cena rovná 22500 Kč.

Náklady ovšem nejsou pouze za samotnou licenci, ale i za lidské zdroje, které zprostředkují nasazení, školení a případná dodatečná nastavení. Tyto úkony jsou zvládnutelné jednou osobou a součinností majitele. Celkem tyto úkony mohou trvat v okolí jednoho, dvou a více pracovních dní. Při standardní osmihodinové pracovní době a ceně 500 Kč/h, se náklady pohybují od 8000 Kč.

Při nasazení je potřeba i součinnost majitele pneuservisu a to hlavně během registrace a školení. Čas, který je na tyto úkony potřeba se pohybuje v rámci hodin.

Pro ostrý provoz je potřeba přesun dat z excelové tabulky do nového systému, to se dá udělat buď automatizovaně nebo ručně. Automatizované řešení ale nezabrání přesunu duplicit, které se v excelové tabulce nachází. Proto je lepší variantou ruční přesun. Během ručního přesunu pravděpodobně není třeba přesouvat celou excelovou tabulku ale stačil by zápis z poslední sezóny, tím se ušetří velké množství času.

Díky jednoduchému rozhraní nového systému může tento ruční přesun trvat jeden až dva týdny. Jedná se o jednoduchou manuální práci, při které je potřeba data kontrolovat, to může pan Lávička zvládnout sám a ušetřit peníze.

Finanční náklady na nasazení systému se pohybují okolo 22500 Kč za licenci pro jednoho uživatele, 8000 Kč za zprostředkování registrace, instalace a školení. 

Celkem kolem \textbf{30500 Kč}.

Časové náklady se pohybují okolo jednoho pracovního dne na registraci a školení a týdne až dvou pro přesun dat. 

Celkem kolem \textbf{48 a více hodin}.
%%%%%%%%%%%%%%%%%%%%%%%%%%%%
\subsection{Náklady na provoz nového IS}
Náklady na provoz se skládají z ročních obnov licence a časových nákladů na práci se systémem. Jak bylo zmíněno v předchozí sekci, náklady na rok provozu činí \textbf{22500 Kč}.

Ačkoli můžeme očekávat zrychlení zadávání informací do nového systému, tak v nejhorším případě by majitel investoval podobné množství času jaké investuje teď, tedy \textbf{50 hodin za sezónu}.
%%%%%%%%%%%%%%%%%%%%%%%%%%%%
\subsection{Náklady na vývoj IS}
Během celého projektu jsem si vedl časovou evidenci prací. Vývoj, od přípravy na první schůzi až po testování, trval 85 hodin. 

Časově nejnáročnější byla implementace, kde bylo potřeba se v systému naučit pracovat. Druhá nejnáročnější činnost, byla vytvoření návrhu, která ale pomohla k rychlejší implementaci.

Při ceně 500 Kč/h za vývojářskou práci, by se celková cena rovnala \textbf{42500 Kč}. 

Díky vývoji v rámci této bakalářské práce se tato cena \textbf{nepromítne} do nákladů pneuservisu. Jedinou investicí pneuservisu do vývoje je čas, který zabrala schůze.
%%%%%%%%%%%%%%%%%%%%%%%%%%%%
\subsection{Přínosy nového IS}
Přínosy nového IS jsou zpřehlednění, rychlejší ukládání a zabezpečeni dat proti poškození. Užitečné je upozornění zákazníku na přezouvací sezónu a objednané přezutí.

Hlavním přínosem je, že nový systém slouží jako stabilní základ pro další rozšíření, která mohou majiteli pneuservisu uvolnit ruce a ušetřit značné množství práce.
