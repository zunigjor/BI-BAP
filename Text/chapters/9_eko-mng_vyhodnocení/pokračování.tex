\section{Pokračování projektu}
%%%%%%%%%%%%%%%%%%%%%%%%%%%%
Pokračování se blízké době týká dvou oblastí, nejdříve představení a případnému nasazení systému, a poté implementaci druhé etapy.
%%%%%%%%%%%%%%%%%%%%%%%%%%%%
\subsection{Představení majiteli}
Implementovaný a otestovaný systém je připraven k představení majiteli pneuservisu. V případě zájmu je možné i předání developerské verze podle sekce o předání \ref{sec:predani}.

V případě, že se pan Lávička pro tento systém rozhodne, je potřeba splnit úkony popsané v sekci \ref{NakladyNasazeni} o nákladech nasazení.
%%%%%%%%%%%%%%%%%%%%%%%%%%%%
\subsection{Implementace druhé etapy} \label{subsec:implementace_druhe_etapy}
Implementaci zákaznických funkcí lze vytvořit pomocí systému \emph{Flows} a vytvoření webových stránek pneuservisu.

Pro implementaci této funkcionality, je klíčový pojem \emph{Inbound Scheduling}, kterým jsou v systému Salesforce označovány rezervace od koncových zákazníků. \cite{SalesforceInboundScheduling}

Cena této implementace se skládá z ceny za vývoj v systému \emph{Flows} a ceny zprovoznění jednoduchých webových stránek.

Hrubým odhadem vývoj této funkcionality může vývojáři zabrat 40 a více hodin. Při ceně 500 Kč/h by se cena vývoje pohybovala od 20000 Kč nahoru.

K této ceně je potřeba přičíst cenu za doménu a hostování webových stránek. 

Cena domény s příponou .cz se pohybuje kolem 200 Kč/rok. Ceny hostování se u českých poskytovatelů liší podle potřeb. Pro tuto aplikaci stačí levnější balíčky, které se cenově pohybují kolem 50 Kč/měsíc. \cite{web4uDomeny}\cite{web4uHosting}\cite{wedosInternet}

Celková cena implementace se může pohybovat od \textbf{20000 Kč}, k této ceně je potřeba přičíst cena za rok provozu, která by se pohybovala kolem \textbf{800 Kč}.
